\section*{Summary of Section 2}
\begin{enumerate}
  \item The group $\bbz/100$ of 2-digit integers sits in a short exact sequence/group extension
  \begin{equation*}
    \begin{tikzcd}
      0 \ar[r] & \bbz/10 \ar[r,"i"] &  \bbz/100  \ar[r,"p"] & \bbz/10 \ar[r] & 0 \\
       & \mbox{tens} \ar[u, equal]  &  \mbox{2-digit numbers}  \ar[u,equal] & \mbox{units}  \ar[u, equal] &
    \end{tikzcd}
  \end{equation*}

  \item Every group extension arises in this manner.
  For an extension of abelian groups,
  \begin{equation*}
    \begin{tikzcd}
      0 \ar[r] & H \ar[r,"i"] &  G  \ar[r, "p"] & K \ar[r] & 0.
    \end{tikzcd}
  \end{equation*}
  every element of $G$ can be uniquely written as $\units{a}\tens{b}$ for some $a \in H$ and $b \in K$.
\end{enumerate}


\vspace{2cm}
\hrule
\vspace{2cm}

% Show that for every element $g \in G$, there exist unique elements $a \in H$ and $ b \in K $ such that $g = \units{a}\tens{b}$.
% \hint{Look at $p(g)$ and $g - [p(g)]\units{0}$.}
\begin{proof}[Solution to Q.\ref{q:existenceOfDigits}]
  We will only prove existence as uniqueness follows by Q.\ref{q:uniquenessOfDigits}.

  Consider the element $g \in G$. Let $b = p(g) \in K$ and let \begin{equation*}
    h = g - \tens{0}\units{b}.
  \end{equation*}
  Applying $p$ to both sides we get
  \begin{align*}
    p(h)
    &= p(g - \tens{0}\units{b}) \\
    &= p(g) - p(\tens{0}\units{b}) && \mbox{ as $p$ is a group homomorphism}\\
    &= b -b \\
    &= 0
  \end{align*}
  Hence, $h \in \ker p$, which implies that $h \in \im i$.
  Hence, $ h = i(a) = \tens{a}\units{0}$ for some $a \in H$, giving us
  \begin{equation*}
    g = h + \tens{0}\units{b} = \tens{a}\units{0} + \tens{0}\units{b} = \tens{a}\units{b}.
  \end{equation*}
\end{proof}
