% !TeX root = index.tex
\section{High School Arithmetic}
We will consider addition in the group $\bbz/100$. Denote the elements of $\bbz/100$ by $\tens{a}\units{b}$ where $a,b \in \set {0, 1, \dots, 9}$.

\subsection{The carry function}
\begin{mdframed}
  \adjustbox{scale=1,center}{%
    \begin{tikzcd}
      &\mbox{Adding 2-digit numbers}
        \ar[ddddl, leftrightarrow]
        % \ar[ddddr, leftrightarrow, end anchor={[xshift=0ex]}]
        &
      \\\\\\\\
      \mbox{2-cocycle condition}
      % \ar[rr, dashed, leftrightarrow]
      & & \text{\color{white} Group extensions}
    \end{tikzcd}
  }
\end{mdframed}


Addition in $\bbz/100$ is defined by the formula
\begin{equation}
  \label{equation:additionFormula}
  \tens{a_1}\units{b_1} + \tens{a_2}\units{b_2}
  =
  \tens{a_1 + a_2 + c(b_1, b_2)}\units{b_1 + b_2}
\end{equation}
where $c: \bbz/10 \times \bbz/10 \rightarrow \bbz/10$ is the ``carry'' function

\begin{qbox}
  Give an explicit definition of the carry function $c$.
\end{qbox}

\begin{qbox}
  Is $c$ a group homomorphism?
\end{qbox}

The binary operation on abelian groups satisfies the following three properties:
\begin{enumerate}
  \item identity: $\qquad x + 0 = x = 0 + x$,
  \item commutativity: $\qquad x+y = y+x$,
  \item associativity: $\qquad (x+y)+z = x + (y+z)$.
\end{enumerate}

\begin{qbox}
  Using Equation \eqref{equation:additionFormula} and the fact that $\bbz/100$ is an abelian group, determine the corresponding identities the function $c$ satisfies?
\end{qbox}
Such a function $c$ has a very fancy name, it is called a
\begin{align*}
  \underbrace{\mathrm{normalized}}_{\mathrm{identity}}
  \underbrace{\mathrm{symmetric}}_{\mathrm{commutativity}}
  \underbrace{\mathrm{2-cocycle}}_{\mathrm{associativity}}.
\end{align*}

\begin{qbox}
  \label{q:cocycleDefinesGroup}
  Conversely, show that if a function $c: \bbz/10 \times \bbz/10 \rightarrow \bbz/10$ is a normalized, symmetric, 2-cocycle then ``defining'' an addition on $\bbz/100$ using Equation \ref{equation:additionFormula} defines an abelian group structure on it.\hint{The only new thing you need to check is that inverses exist.}
\end{qbox}

% \begin{qbox}
%    Let $0 \le k < 10$ be an integer. Check that \begin{align*}
%     c_k: \bbz/10 \times \bbz/10 &\rightarrow \bbz/10\\
%     c_k(b_1,b_2) &=
%       \begin{cases}
%         0 & \mbox{if } b_1 + b_2 < 10, \\
%         k & \mbox{if } b_1 + b_2 \ge 10.
%       \end{cases}
%   \end{align*} is also a normalized symmetric 2-cocycle.
% \end{qbox}

% Let $N_k = \set{ \tens{a} \units{b}}$ be the set of two-digit numbers with addition defined as
% \begin{align*}
%   +_k: N_k \times N_k
%     &\longrightarrow N_k, \\
%   \tens{a_1}\units{b_1} +_k \tens{a_2}\units{b_2}
%     &=
%     \tens{a_1 + a_2 + c_k(b_1, b_2)}\units{b_1 + b_2}.
% \end{align*}
%
% By Q.\ref{q:cocycleDefinesGroup}, $(N_k, +_k)$ is an abelian group.

Using 2-cocycles, it is possible to create some very exotic ``carry functions''.
\begin{qbox}
  \label{q:symmetricCocycles}
  Come up with other examples of normalized, symmetric, 2-cocycles $c:\bbz/10 \times \bbz/10 \rightarrow \bbz/10$.
\end{qbox}

There are exactly 4 isomorphism classes of abelian groups of order 100:
  \begin{equation*}
    \bbz/100,\quad \bbz/50 \times \bbz/2, \quad \bbz/20 \times \bbz/5, \quad \bbz/10 \times \bbz/10.
  \end{equation*}

\begin{qbox}
  Each of the examples of carry functions $c$ you found in Q.\ref{q:symmetricCocycles} defines an abelian group structure on the set of 2-digit numbers. Find its isomorphism class.
\end{qbox}

\begin{qbox}[Bonus Question]
  Can you find all the normalized, symmetric, 2-cocycles $c:\bbz/10 \times \bbz/10 \rightarrow \bbz/10$? (I do not know any way to answer this question without using group cohomology computations. It would be amazing if you could solve this problem by some elementary methods.)
\end{qbox}
