\newpage
\section{Group extensions}
Today our goal is to understand how addition of 2-digit numbers is equivalent to another abstract construction in group theory called group extensions.
Using this interpretation and some facts from group cohomology, we will eventually be able to answer the question,
\begin{q*}
  What are all the normalized, symmetric, 2-cocycles $c:\bbz/10 \times \bbz/10 \rightarrow \bbz/10$?
\end{q*}
\begin{mdframed}
  \adjustbox{scale=1,center}{%
    \begin{tikzcd}
      &\mbox{Adding 2-digit numbers}
        \ar[ddddr, leftrightarrow]
        &
      \\\\\\\\
      \mbox{\color{white}2-cocycle condition}
      & & \text{Group extensions}
    \end{tikzcd}
  }
\end{mdframed}





\subsection{Review: Group theory}
All our groups will be abelian unless otherwise specified.

A map between abelian groups $\varphi: G_1 \longrightarrow G_2$ is a group homomorphism if it satisfies
\begin{align*}
  \varphi(a + b) = \varphi(a) + \varphi(b)
\end{align*}

\begin{qbox}(Practice problem)
  \label{q:groupHomsExamples}
  Find all the group homomorphisms
  \begin{enumerate}
    \item $\bbz \longrightarrow \bbz$,
    \item $\bbz \longrightarrow \bbz/n$,
    \item $\bbz/n \longrightarrow \bbz$,
    \item $\bbz/n \longrightarrow \bbz/n^2$,
    \item $\bbz/n^2 \longrightarrow \bbz/n$,
  \end{enumerate}
  where $n$ is a positive integer.
\end{qbox}

\begin{qbox}
  Let $M$ be an abelian group. Describe all the group homomorphisms $ \bbz/n \longrightarrow M$, where $n$ is a positive integer.
  (This set will show up again when we discuss group cohomology.)
\end{qbox}

\begin{definition}
  The \emph{kernel} of a group homomorphism  $\varphi: G_1 \longrightarrow G_2$ is the set of elements $g \in G_1$ such that $\varphi(g) = 0$.
  \begin{align*}
    \ker \varphi = \set{ g \in G_1 : \varphi(g) = 0}
  \end{align*}
\end{definition}

\begin{definition}
  The \emph{image} of a group homomorphism  $\varphi: G_1 \longrightarrow G_2$ is the set of elements $\varphi(g) \in G_2$ where $g \in G_1$.
  \begin{align*}
    \im \varphi = \set{ \varphi(g) : g \in G_1}
  \end{align*}
\end{definition}


\begin{qbox}
  Find the kernel and image of the group homomorphisms you found in Q.\ref{q:groupHomsExamples}.
\end{qbox}

\begin{qbox}
  Which of the group homomorphisms in Q.\ref{q:groupHomsExamples} are
  \begin{enumerate}
    \item injective (=one-to-one)?
    \item surjective (=onto)?
    \item isomorphisms (=one-to-one and onto)?
  \end{enumerate}
\end{qbox}

\subsection*{Optional practice problems}

\begin{qbox}
  Show that the image of a group homomorphism  $\varphi: G_1 \longrightarrow G_2$ is a subgroup of $G_2$.
\end{qbox}

\begin{qbox}
  Show that the kernel of a group homomorphism  $\varphi: G_1 \longrightarrow G_2$ is a subgroup of $G_1$.
\end{qbox}

\begin{qbox}
  Show that for a group homomorphism $\varphi: G_1 \longrightarrow G_2$ we have $G_1 / \ker \varphi \cong \im \varphi$.
\end{qbox}














\newpage
\subsection{Group extensions}

\begin{qbox}
  \label{q:inclusionMaps}
  Consider the two inclusion maps,
  \begin{align*}
    i_u: \bbz/10 &\longrightarrow \bbz/100  &&& i_t: \bbz/10 & \longrightarrow \bbz/100 \\
    b &\longmapsto \tens{0}\units{b} &&& a &\longmapsto \tens{a}\units{0}
  \end{align*}
  Which of these two maps is a group homomorphism?
\end{qbox}

\begin{qbox}
  \label{q:projectionMaps}
  Consider the two projection maps,
  \begin{align*}
    p_u: \bbz/100 &\longrightarrow \bbz/10 &&& p_t: \bbz/100 &\longrightarrow \bbz/10 \\
    \tens{a}\units{b} &\longmapsto {b} &&& \tens{a}\units{b} &\longmapsto a
  \end{align*}
  Which of these two maps is a group homomorphism?
\end{qbox}


\begin{qbox}
  \label{q:SES1}
  For the $i$ and $p$ in Questions \ref{q:inclusionMaps} and \ref{q:projectionMaps} that are group homomorphisms, check that
  \begin{enumerate}
    \item $i$ is injective,
    \item $p$ is surjective,
    \item $\im i = \ker p$.
  \end{enumerate}
\end{qbox}

\begin{definition}
  An \emph{extension} of a group $K$ by $H$ is a group $G$ along with a pair of maps
  \begin{align*}
    i: H \longrightarrow G && p: G \longrightarrow K
  \end{align*}
  such that
  \begin{enumerate}
    \item $i$ is injective,
    \item $p$ is surjective,
    \item $\im i = \ker p$.
  \end{enumerate}
\end{definition}

This is often written as a \emph{short exact sequence}
\begin{equation*}
  \begin{tikzcd}
    0 \ar[r] & H \ar[r] &  G  \ar[r] & K \ar[r] & 0
  \end{tikzcd}
\end{equation*}















\newpage
Q.\ref{q:SES1} is saying that the following is a SES
\begin{equation*}
  \begin{tikzcd}
    0 \ar[r] & \bbz/10 \ar[r,"i"] &  \bbz/100  \ar[r,"p"] & \bbz/10 \ar[r] & 0 \\
     & \mbox{tens} \ar[u, equal]  &  \mbox{2-digit numbers}  \ar[u,equal] & \mbox{units}  \ar[u, equal] &
  \end{tikzcd}
\end{equation*}

In fact, every group extension arises in this manner.
Consider an extension of abelian groups,
\begin{equation*}
  \begin{tikzcd}
    0 \ar[r] & H \ar[r,"i"] &  G  \ar[r, "p"] & K \ar[r] & 0.
  \end{tikzcd}
\end{equation*}
i.e.
\begin{enumerate}
  \item $i$ is injective,
  \item $p$ is surjective,
  \item $\im i = \ker p$.
\end{enumerate}

For an element $a \in H$, denote by $\tens{a}\units{0} \in G$ the element $i(a)$.
For an element $b \in K$, let $\tens{0}\units{b}$ be \emph{some} element in $G$ such that $p(\tens{0}\units{b}) = b$.
Define $\tens{a} \units{b}$ to be the element $\tens{a}\units{0} + \tens{0}\units{b}$ in $G$.

\begin{qbox}
  Show that $p(\tens{a}\units{0}) = 0$ for any $a \in H$.
\end{qbox}

% \begin{qbox}
%   % \begin{enumerate}
%   %   \item
%     Check that $\tens{a_1}\units{0} + \tens{a_2}\units{0} = \tens{a_1 + a_2}\units{0}$.
%     % \item Is it true that $\tens{0}\units{b_1} + \tens{0}\units{b_2} = \tens{0}\units{b_1 + b_2}$?
%   % \end{enumerate}
% \end{qbox}

\begin{qbox}
  \label{q:uniquenessOfDigits}
  Let $a_1$, $a_2$ be elements in $H$ and let $b_1$, $b_2$ be elements in $K$, show that\hint{Apply $p$ to both sides of $[a_1][b_1] = [a_2][b_2]$.}
\begin{align*}
    [a_1][b_1] = [a_2][b_2] \implies a_1 = a_2 \mbox{ and } b_1 = b_2.
\end{align*}
\end{qbox}

\begin{qbox}
  \label{q:existenceOfDigits}
  Show that for every element $g \in G$, there exist unique elements $a \in H$ and $ b \in K $ such that $g = \units{a}\tens{b}$.
  \hint{Look at $p(g)$ and $g - \tens{p(g)}\units{0}$.}
\end{qbox}
