\section*{Summary of Section 1}
\begin{itemize}
  \item Addition in $\bbz/100$ is defined by the formula
  \begin{equation*}
    \tens{a_1}\units{b_1} + \tens{a_2}\units{b_2}
    =
    \tens{a_1 + a_2 + c(b_1, b_2)}\units{b_1 + b_2}
  \end{equation*}
  where $c: \bbz/10 \times \bbz/10 \rightarrow \bbz/10$ is the ``carry'' function.
  Unravelling the axioms of abelian groups we see that $c$ satisfies the following three identities:
  \begin{enumerate}
    \item $c(b_1, 0) = 0 = c(0,b_1)$,
    \item $c(b_1, b_2) = c(b_2, b_1)$,
    \item $c(b_1, b_2) + c(b_1 + b_2, b_3) = c(b_1 , b_2 + b_3) + c(b_2, b_3)$.
  \end{enumerate}
  Such a function $c$ is called a \emph{normalized, symmetric, 2-cocycle}.

  \item We now flip the tables and ``define'' an addition on the set of 2-digit numbers by the formula $\tens{a_1}\units{b_1} + \tens{a_2}\units{b_2}
  =
  \tens{a_1 + a_2 + c(b_1, b_2)}\units{b_1 + b_2}$ where $c$ is any normalized, symmetric, 2-cocycle.

  \item Examples:
    \begin{enumerate}
      \item $c(b_1, b_2) = \left \lfloor \dfrac{b_1 + b_2}{10}  \right \rfloor $ defines the standard addition on $\bbz/100$.
      \item $c(b_1, b_2) = 0$ defines the addition in which the set of 2-digit numbers becomes $\bbz/10 \times \bbz/10$.
      \item $c(b_1, b_2) = k\left \lfloor \dfrac{b_1 + b_2}{10}  \right \rfloor$ for any integer $k$ defines an addition on the set of 2-digit numbers, the isomorphism class of the resulting abelian group depends on $k \mod 10$.
      \item $c(b_1, b_2) = b_1 b_2$ defines an addition on the set of 2-digit numbers, and resulting group is $\bbz/20 \times \bbz/5$.
    \end{enumerate}
\end{itemize}


There are exactly 4 isomorphism classes of abelian groups of order 100:
  \begin{equation*}
    \bbz/100,\quad \bbz/50 \times \bbz/2, \quad \bbz/20 \times \bbz/5, \quad \bbz/10 \times \bbz/10.
  \end{equation*}
\begin{q*}
  How do we know which abelian group is being created by using a particular addition?
\end{q*}
\begin{proof}[Answer]
  The four groups $\bbz/100$, $\bbz/50 \times \bbz/2$, $\bbz/20 \times \bbz/5$, $\bbz/10$ can be differentiated in the following way:
  $\bbz/100$ contains an element of order 100,
  $\bbz/50 \times \bbz/2$ contains an element of order 50 but not of order 100,
  $\bbz/20 \times \bbz/5$ contains an element of order 20 but not of order 100,
  the order of every element of $\bbz/10 \times \bbz/10$ is at most 10.
\end{proof}
